
\subsection{Summed McCabe Complexity}
McCabe (cyclomatic) complexity \cite{McCabe:SoftwareComplexity:76} measures\footnote{http://parisc-linux.org/~bame/pmccabe/} the number of independent paths through a given function.  This is a function of the number of loops/conditionals and the like.  This quantity is unbounded/non-normalized.  More functions will yield a higher value.


\begin{tabular}{l r}
{gnu/usr.bin/gcc/gcc/recog.c} & 755 \\
{gnu/gcc/gcc/recog.c} & 754 \\
{gnu/usr.bin/perl/hv.c} & 651 \\
{gnu/gcc/gcc/cp/decl2.c} & 622 \\
{usr.sbin/bgpd/rde.c} & 618 \\
{sys/dev/audio.c} & 604 \\
{gnu/usr.bin/gcc/gcc/config/d30v/d30v.c} & 602 \\
{gnu/gcc/gcc/config/cris/cris.c} & 593 \\
{gnu/usr.bin/gcc/gcc/config/mcore/mcore.c} & 583 \\
{gnu/usr.bin/binutils-2.17/gas/config/tc-sh64.c} & 568 \\
{gnu/usr.bin/binutils/gas/config/tc-sh64.c} & 562 \\
{gnu/gcc/gcc/dbxout.c} & 518 \\
{gnu/usr.bin/binutils-2.17/bfd/elf32-s390.c} & 497 \\
{gnu/usr.bin/gcc/gcc/java/expr.c} & 429 \\
{sys/dev/pci/if-oce.c} & 413 \\
{sys/dev/ic/mpi.c} & 390 \\
{gnu/usr.bin/binutils-2.17/binutils/dlltool.c} & 345 \\
{sys/dev/pci/if-san-xilinx.c} & 343 \\
{sys/arch/arm/arm/pmap7.c} & 329 \\
{lib/libpcap/gencode.c} & 300 \\
{sys/dev/isa/gus.c} & 286 \\
{usr.sbin/nginx/src/http/modules/ngx-http-mp4-module.c} & 246 \\
{gnu/usr.bin/binutils/binutils/rcparse.c} & 127 \\
{gnu/usr.bin/binutils-2.17/opcodes/xc16x-desc.c} & 45 \\
\end{tabular}


\subsection{Averaged McCabe Complexity}
\label{McCabeAveraged}

Since the McCabe complexity is unbounded, we divide by the number of functions to get an average value for the entire file.  This table compares this value to the H-Value given by the DFA analysis.


\begin{tabular}{l r r}
\textbf{Filename} & \textbf{H-Value} & \textbf{McCabe Average} \\
gnu/usr.bin/gcc/gcc/recog.c &    0.5966 &  11.98 \\
gnu/gcc/gcc/recog.c          &   0.59523 & 11.42  \\
gnu/usr.bin/perl/hv.c        &   0.69928 & 10.02 \\
usr.sbin/bgpd/rde.c          &   0.61891 & 10.13 \\
sys/dev/audio.c               &  0.58739  & 8.27 \\
gnu/gcc/gcc/config/cris/cris.c & 0.58113 & 9.72 \\
gnu/usr.bin/gcc/gcc/config/mcore/mcore.c & 0.67327 & 7.02 \\
gnu/gcc/gcc/dbxout.c         &   0.59439 & 7.85   \\
gnu/usr.bin/gcc/gcc/java/expr.c & 0.62566 & 5.29 \\
sys/dev/ic/mpi.c             &   0.66608 & 5.34      \\
sys/dev/pci/if-san-xilinx.c   &  0.56351 &  NAN \\
sys/arch/arm/arm/pmap7.c    &    0.5988 &  4.76 \\
\end{tabular}

